\noindent The value of streaming systems like Spark and Storm is derived from the timeliness of the results. Less amount of time to process the incoming data means faster reaction from either users or machines. Nowadays, different streaming systems provide different end-to-end time guarantees, and these guarantees are to a large extent a result of each systems architecture. Spark Streaming, uses a general batch-processing to compute over streaming data. Storm allows the construction of topologies on which data flows to be processed. Spark Streaming is known to provide latencies on the order of seconds while Storm on the order of low milliseconds. In this paper we investigate the approach taken by Spark Streaming and show that despite its more general architecture it can support task short task latencies. To do so, we tackled two problems. The first has to do with overheads related to launching tasks and communicating between nodes. The second has to do with the poor scalability performance when the Spark is subject to a high number of tasks being scheduled. We show that with these two contributions Spark Streaming can support low-latency tasks and scale to demanding stream workloads.


%Large-scale data processing systems such as Spark and Hadoop MapReduce were developed for clusters built on commodity hardware. In this paper, we argue that current market trends invalidate some of the assumptions made by those systems, and we should therefore revisit the designs and architectures derived from them. For example, newer technologies including InfiniBand offer faster network connectivity, and advances in SSDs continue to drive down the cost of fast non-volatile memory. One new architecture that explores these new technologies is Firebox, a hardware building block for future Warehouse-Scale Computers (WSCs). We use Firebox in our studies, and evaluate the behavior changes of large-scale systems under the new settings. In particular, we measure the performance of Spark and applications in the Spark ecosystem under specific workloads, and critique the validity of its underlying assumptions and design decisions.
